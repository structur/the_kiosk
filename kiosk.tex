\documentclass[20 pt,twoside,extrafontsizes,final]{memoir}
\usepackage{geometry}

\textwidth =6.5in
\setlrmargins{*}{*}{*}
\setulmargins{*}{*}{2}
\checkandfixthelayout

\font\HUGE = cmr10 at 80pt
\font\sl = cmsl10 at 17pt
\begin{document}
\title{The Kiosk}
\author{Bruce Sterling}
\date{}


\begin{titlingpage}\begin{vplace}[0.6]
\vspace{40pt}
\centerline{\HUGE The Kiosk}\bigskip\
\centerline{\Huge Bruce Sterling}
\end{vplace}
\begin{flushright}
Typset by structur
\end{flushright}\end{titlingpage}

\frontmatter

\centerline{\sl About the Author}\smallskip
{\scriptsize Bruce Sterling reports that in 2005 he received an honorary doctorate from Art Center College of Design in California. He also teaches sometimes at the European Graduate School in Switzerland. So nowadays, he says, ``I'm  `Professor-Doktor Sterling,' but I didn't even shine my shoes or get a haircut.'' He also mentions that he's hard at work on a new sf novel. His new story is a fine work of speculative fiction.}\medskip

\centerline{\sl About the Book}\smallskip
{\openup 1em\scriptsize This is a book about 3D printing, and the promises of emancipation that are read into a technology which, rather than reclaims the means of production, sublimates it. The dream of 3D printing is that we might be free of rigid, elite-owned production infrastructures, and construct what we want, when we want it-- as long as we can design it, or another will share their design.  This free and open model of production echoes the printing press, which allowed knowledge to widely disseminated at far less cost. In time, perhaps 3D printing will provide comparable benefits to people. In this book, the process of bookbinding meets 3D printing- using traditional techniques of binding for the text block, but covered by a materialized digital model. It is typeset entirely in \LaTeX, and the cover is printed in MakerBot PLA. Thanks to Bruce Sterling for offering this story under a Creative Commons License. }\bigskip

\centerline{{$\bigtriangleup\bigtriangledown\bigtriangleup\bigtriangledown\bigtriangleup$}}
\vfil\eject

\mainmatter


\chapter*{\centerline{I}}

\normalsize The fabrikator was ugly, noisy, a fire hazard, and it smelled. Borislav got it for the kids in the neighborhood.

One snowy morning, in his work gloves, long coat, and fur hat, he loudly power-sawed through the wall of his kiosk. He duct-taped and stapled the fabrikator into place.
 
The neighborhood kids caught on instantly. His new venture was a big hit. The fabrikator made little plastic toys from 3-D computer models. After a week, the fab's dirt-cheap toys literally turned into dirt. The fabbed toys just crumbled away, into a waxy, non-toxic substance that the smaller kids tended to chew.

Borislav had naturally figured that the brief lifetime of these toys might discourage the kids from buying them. This just wasn't so. This wasn't a bug: this was a feature. Every day after school, an eager gang of kids clustered around Borislav's green kiosk. They slapped down their tinny pocket change with mittened hands. Then they exulted, quarrelled, and sometimes even punched each other over the shining fab-cards.

The happy kid would stick the fab-card (adorned with some glossily fraudulent pic of the toy) into the fabrikator's slot. After a hot, deeply exciting moment of hissing, spraying, and stinking, the fab would burp up a freshly minted dinosaur, baby doll, or toy fireman.

Foot traffic always brought foot traffic. The grownups slowed as they crunched the snowy street. They cast an eye at the many temptations ranked behind Borislav's windows. Then they would impulse-buy. A football scarf, maybe. A pack of tissues for a sneezy nose.

Once again he was ahead of the game: the only kiosk in town with a fabrikator. The fabrikator spoke to him as a veteran street merchant. Yes, it definitely meant something that those rowdy kids were so eager to buy toys that fell apart and turned to dirt. Any kiosk was all about high-volume repeat business. The stick of gum. The candy bar. The cheap, last-minute bottle-of-booze. The glittery souvenir keychain that tourists would never use for any purpose whatsoever. These objects were the very stuff of a kiosk's life.

Those colored plastic cards with the 3-D models\dots the cards had potential. The older kids were already collecting the cards: not the toys that the cards made, but the cards themselves.
And now, this very day, from where he sat in his usual street-cockpit behind his walls of angled glass, Borislav had taken the next logical step. He offered the kids ultra-glossy, overpriced, collector cards that could not and would not make toys. And of course--there was definitely logic here--the kids were going nuts for that business model. He had sold a hundred of them.
Kids, by the nature of kids, weren't burdened with a lot of cash. Taking their money was not his real goal. What the kids brought to his kiosk was what kids had to give him--futurity. Their little churn of street energy--that was the symptom of something bigger, just over the horizon. He didn't have a word for that yet, but he could feel it, in the way he felt a coming thunderstorm inside his aching leg. Futurity might bring a man money. Money never saved a man with no future.

\vfill


\centerline{{$\bigtriangleup\bigtriangledown\bigtriangleup\bigtriangledown\bigtriangleup$}}

\chapter*{\centerline{II}}
Dr. Grootjans had a jaw like a horse, a round blue pillbox of a hat, and a stiff winter coat that could likely stop gunfire. She carried a big European shopping wand.

Ace was acting as her official street-guide, an unusual situation, since Ace was the local gangster.
``Madame,'' Ace told her, ``this is the finest kiosk in the city. Boots here is our philosopher of kiosks. Boots has a fabrikator! He even has a water fountain!''

Dr. Grootjans carefully photographed the water fountain's copper pipe, plastic splash basin, and disposable paper pop-out cups. 

``Did my guide just call you `Boots?' '' she said. ``Boots as in footgear?''

``Everybody calls me that.''

Dr. Grootjans patted her translation earpiece, looking pleased. ``This water-fountain is the exhaust from your fuel cell.''

Borislav rubbed his mustache. ``When I first built my kiosk here, the people had no running water.''

Dr. Grootjans waved her digital wand over his selections of panty-hose. She photographed the rusty bolts that fixed his kiosk to the broken pavement. She took particular interest in his kiosk's peaked roof.

People often met their friends and lovers at Borislav's kiosk, because his towering satellite dish was so easy to spot. With its painted plywood base and showy fringes of snipped copper, the dish looked fit for a minaret.

``Please try on this pretty necklace, madame! Made by a fine artist, she lives right up the street. Very famous. Artistic. Valuable. Regional. Handmade!''

``Thank you, I will. Your shop is a fine example of the local small-to-micro regional enterprise. I must make extensive acquisitions for full study by the Parliamentary committee.''

Borislav swiftly handed over a sheet of flimsy. Ace peeled off a gaping plastic bag and commenced to fill it with candy bars, placemats, hand-knitted socks, peasant dolls in vests and angular headdresses, and religious-war press-on tattoos.

``He has such variety, madame! Such unusual goods!''

Borislav leaned forward through his cash window, so as to keep Dr. Grootjans engaged as Ace crammed her bags. ``Madame, I don't care to boast about my modest local wares... Because whatever I sell is due to the people! You see, dear doctor-madame, every object desired by these colorful local people has a folk-tale to tell us\dots''

Dr. Grootjan's pillbox hat rose as she lifted her brows. ``A folk tale, did you say?''

``Yes, it's the people's poetry of commerce! Certain products appear ... the products flow through my kiosk ... I present them pleasingly, as best I can\dots Then, the people buy them, or they just don't buy!''

Dr. Grootjans expertly flapped open a third shopping bag. ``An itemized catalog of all your goods would be of great interest to my study committee.''

Borislav put his hat on.

Dr. Grootjans bored in. ``I need the complete, digital inventory of your merchandise. The working file of the full contents of your store. Your commercial records from the past five years will be useful in spotting local consumer trends.''

Borislav gazed around his thickly packed shelving. ``You mean you want a list of everything I sell in here? Who would ever find the time?''

``It's simple! You must have heard of the European Unified Electronic Product Coding System.'' Dr. Grootjans tucked the shopping wand into her canvas purse, which bore an imperial logo of thirty-five golden stars in a widening spiral. ``I have a smart-ink brochure here which displays in your local language. Yes, here it is:
`A Partial Introduction to EU-EPCS Regulatory Adoption Procedures.'''

Borislav refused her busily flashing inkware. ``Oh yes, word gets around about that electric barcoding nonsense! Those fancy radio-ID stickers of yours. Yes, yes, I'm sure those things are just fine for rich foreign people with shopping-wands!''

``Sir, if you sensibly deployed this electronic tracking system, you could keep complete, real-time records of all your merchandise. Then you would know exactly what's selling, and not. You could fully optimize your product flow, reduce waste, maximize your profit, and benefit the environment through reduced consumption.''

Borislav stared at her. ``You've given this speech before, haven't you?''

``Of course I have! It's a critical policy issue! The modern Internet-of-Things authenticates goods, reduces spoilage, and expedites secure cross-border shipping!''

``Listen, madame doctor: your fancy bookkeeping won't help me if I don't know the soul of the people! I have a little kiosk! I never compete with those big, faceless mall stores! If you want that sort of thing, go shop in your five-star hotel!''

Dr. Grootjans lowered her sturdy purse and her sharp face softened into lines of piety. ``I don't mean to violate your quaint local value system\dots Of course we fully respect your cultural differences\dots Although there will be many tangible benefits when your regime fully harmonizes with European procedures.''

```My regime,' is it? Ha!'' Borislav thumped the hollow floor of his kiosk with his cane. ``This stupid regime crashed all their government computers! Along with crashing our currency, I might add! Those crooks couldn't run that fancy system of yours in a thousand years!''

``A comprehensive debate on this issue would be fascinating!'' Dr. Grootjans waited expectantly, but, to her disappointment, no such debate followed. ``Time presses,'' she told him at last. ``May I raise the subject of a complete acquisition?''

Borislav shrugged. ``I never argue with a lady or a paying customer. Just tell what you want.''
Dr. Grootjans sketched the air with her starry wand. ``This portable shelter would fit onto an embassy truck.''

``Are you telling me that you want to buy my entire kiosk?''

``I'm advancing that option now, yes.''

``What a scandal! Sell you my kiosk? The people would never forgive me!''

``Kiosks are just temporary structures. I can see your business is improving. Why not open a permanent retail store? Start over in a new, more stable condition. Then you'd see how simple and easy regulatory adoption can be!''

Ace swung a heavily laden shopping bag from hand to hand. ``Madame, be reasonable! This street just can't be the same without this kiosk!''

``You do have severe difficulties with inventory management. So, I will put a down-payment on the contents of your store. Then,'' she turned to Ace, ``I will hire you as the inventory consultant. We will need every object named, priced and cataloged. As soon as possible. Please.''

\bigskip\centerline{{$\bigtriangleup\bigtriangledown\bigtriangleup\bigtriangledown\bigtriangleup$}}\bigskip
Borislav lived with his mother on the ground floor of a local apartment building. This saved him trouble with his bad leg. When he limped through the door, his mother was doing her nails at the kitchen table, with her hair in curlers and her feet in a sizzling foot-bath.

Borislav sniffed at the stew, then set his cane aside and sat in a plastic chair.
``Mama dear, heaven knows we've seen our share of bad times.''

``You're late tonight, poor boy! What ails you?''

``Mama, I just sold my entire stock! Everything in the kiosk! All sold, at one great swoop! For hard currency, too!'' Borislav reached into the pocket of his long coat. ``This is the best business day I've ever had!''

``Really?''

``Yes! It's fantastic! Ace really came through for me--he brought his useful European idiot, and she bought the whole works! Look, I've saved just one special item, just for you.''

She raised her glasses on a neck-chain. ``Are these new fabbing cards?''

``No, Mama. These fine souvenir playing cards feature all the stars from your favorite Mexican soap operas. These are the originals, still in their wrapper! That's authentic cellophane!''

His mother blew on her wet scarlet nails, not daring to touch her prize.
``Cellophane! Your father would be so proud!''

``You're going to use those cards very soon, Mama. Your Saint's-Day is coming up. We're going to have a big bridge party for all your girlfriends. The boys at the Three Cats are going to cater it! You won't have to lift one pretty finger!''

Her mascaraed eyes grew wide. ``Can we afford that?''

``I've already arranged it! I talked to Mirko who runs the Three Cats, and I hired Mirko's weird gay brother-in-law to decorate that empty flat upstairs. You know--that flat nobody wants to rent, where that mob guy shot himself. When your old girls see how we've done that place up, word will get around. We'll have new tenants in no time!''

``You're really fixing the haunted flat, son?''

Borislav changed his winter boots for his wooly house slippers. ``That's right, Mama. That haunted flat is gonna be a nice little earner.''

``It's got a ghost in it.''

``Not anymore, it doesn't. From now on, we're gonna call that place\dots what was that French word he used?--we'll call it the `atelier!'''

``The `atelier!' Really! My heart's all a-flutter!''

Borislav poured his mother a stiff shot of her favorite digestive. ``Mama, maybe this news seems sudden, but I've been expecting this. Business has been looking up. Real life is changing, for the real people in this world. The people like us!'' 

Borislav poured himself a brimming cup of flavored yogurt. ``Those fancy foreigners, they don't even understand what the people are doing here today!''

``I don't understand all this men's political talk.''

``Well, I can see it on their faces. I know what the people want. The people\dots They want a new life.''

She rose from her chair, shaking a little. ``I'll heat up your stew. It's getting so late.''

``Listen to me, Mama. Don't be afraid of what I say. I promise you something. You're going to die on silk sheets. That's what this means. That's what I'm telling you. There's gonna be a handsome priest at your bedside, and the oil and the holy water, just like you always wanted. A big granite headstone for you, Mama, with big golden letters.''

As he ate his stew, she began to weep with joy.

\bigskip\centerline{{$\bigtriangleup\bigtriangledown\bigtriangleup\bigtriangledown\bigtriangleup$}}\bigskip

After supper, Borislav ignored his mother's usual nagging about his lack of a wife. He limped down to the local sports bar for some serious drinking. Borislav didn't drink much anymore, because the kiosk scanned him whenever he sat inside it. It used a cheap superconductive loop, woven through the fiberboard walls. The loop's magnetism flowed through his body, revealing his bones and tissues on his laptop screen. Then the scanner compared the state of his body to its records of past days, and it coughed up a medical report.

This machine was a cheap, pirated copy of some hospital's fancy medical scanner. There had been some trouble in spreading that technology, but with the collapse of public health systems, people had to take some matters into their own hands. Borislav's health report was not cheery. He had plaque in major arteries. He had some seed-pearl kidney-stones. His teeth needed attention. Worst of all, his right leg had been wrecked by a land mine. The shinbone had healed with the passing years, but it had healed badly.

The foot below his old wound had bad circulation. Age was gripping his body, visible right there on the screen. Though he could witness himself growing old, there wasn't much he could do about that. Except, that is, for his drinking. Borislav had been fueled by booze his entire adult life, but the alcohol's damage was visibly spreading through his organs. Lying to himself about that obvious fact simply made him feel like a fool. So, nowadays, he drank a liter of yogurt a day, chased with eco-correct paper cartons of multivitamin fruit juices, European-approved, licensed, and fully patented. He did that grumpily and he resented it deeply, but could see on the screen that it was improving his health.

So, no more limping, pitching and staggering, poetically numbed, down the midnight streets. Except for special occasions, that is. Occasions like this one. Borislav had a thoughtful look around the dimly lit haunt of the old Homeland Sports Bar. So many familiar faces lurking in here--his daily customers, most of them. The men were bundled up for winter. Their faces were rugged and lined. Shaving and bathing were not big priorities for them. They were also drunk. But the men wore new, delicately tinted glasses.

They had nice haircuts. Some had capped their teeth. The people were prospering.
Ace sat at his favorite table, wearing a white cashmere scarf, a tailored jacket and a dandified beret. Five years earlier, Ace would have had his butt royally kicked for showing up at the Homeland Sports Bar dressed like an Italian. But the times were changing at the Homeland.
Bracing himself with his cane, Borislav settled into a torn chair beneath a gaudy flat-screen display, where the Polish football team were making fools of the Dutch.

``So, Ace, you got it delivered?''

Ace nodded. ``Over at the embassy, they are weighing, tagging and analyzing every single thing you sell.''

``That old broad's not as stupid as she acts, you know.''

``I know that. But when she saw that cheese-grater that can chop glass. The tiling caulk that was also a dessert!'' Ace half-choked on the local cognac. ``And the skull adjuster! God in heaven!''

Borislav scowled. ``That skull adjuster is a great product! It'll chase a hangover away--'' Borislav snapped his fingers loudly ``--just like that!''

The waitress hurried over. She was a foreign girl who barely spoke the language, but there were a lot of such girls in town lately. Borislav pointed at Ace's drink. ``One of those, missie, and keep `em coming.''

``That skull squeezer of yours is a torture device. It's weird, it's nutty. It's not even made by human beings.''

``So what? So it needs a better name and a nicer label. `The Craniette,' some nice brand-name. Manufacture it in pink. Emboss some flowers on it.''

``Women will never squeeze their skulls with that crazy thing.''

``Oh yes they will. Not old women from old Europe, no. But some will. Because I've seen them do it. I sold ten of those! The people want it!''

``You're always going on about `what the people want.' ''

``Well, that's it! That's our regional competitive advantage! The people who live here, they have a very special relationship to the market economy.'' Borislav's drink arrived. He downed his shot.

``The people here,'' he said, ``they're used to seeing markets wreck their lives and turn everything upside down. That's why we're finally the ones setting the hot new trends in today's world, while the Europeans are trying to catch up with us! These people here, they love the new commercial products with no human origin!''

``Dr. Grootjans stared at that thing like it had come from Mars.''

``Ace, the free market always makes sense--once you get to know how it works. You must have heard of the `invisible hand of the market.' ''

Ace downed his cognac and looked skeptical.

``The invisible hand--that's what gives us products like the skull-squeezer. That's easy to understand.''

``No it isn't. Why would the invisible hand squeeze people's heads?''

``Because it's a search engine! It's mining the market data for new opportunities. The bigger the market, the more it tries to break in by automatically generating new products. And that headache-pill market, that's one of the world's biggest markets!''

Ace scratched under his armpit holster. ``How big is that market, the world market for headaches?''

``It's huge! Every convenience store sells painkillers. Little packets of two and three pills, with big price markups. What are those pills all about? The needs and wants of the people!''

``Miserable people?''

``Exactly! People who hate their jobs, bitter people who hate their wives and husbands. The market for misery is always huge.'' Borislav knocked back another drink. ``I'm talking too much tonight.''

``Boots, I need you to talk to me. I just made more money for less work than I have in a long time. Now I'm even on salary inside a foreign embassy. This situation's getting serious. I need to know the philosophy--how an invisible hand makes real things. I gotta figure that out before the Europeans do.''

``It's a market search engine for an Internet-of-Things.''

Ace lifted and splayed his fingers. ``Look, tell me something I can get my hands on. You know. Something that a man can steal.''

``Say you type two words at random: any two words. Type those two words into an Internet search engine. What happens?''

Ace twirled his shot glass. ``Well, a search engine always hits on something, that's for sure. Something stupid, maybe, but always something.''

``That's right. Now imagine you put two products into a search engine for things. So let's say it tries to sort and mix together ... a parachute and a pair of shoes. What do you get from that kind of search?''

Ace thought it over. ``I get it. You get a shoe that blows up a plane.''

Borislav shook his head. ``No, no. See, that is your problem right there. You're in the racket, you're a fixer. So you just don't think commercially.''

``How can I outthink a machine like that?''

``You're doing it right now, Ace. Search engines have no ideas, no philosophy. They never think at all. Only people think and create ideas. Search engines are just programmed to search through what people want. Then they just mix, and match, and spit up some results. Endless results. Those results don't matter, though, unless the people want them. And here, the people want them!''

The waitress brought a bottle, peppered sauerkraut, and a leathery loaf of bread. Ace watched her hips sway as she left. ``Well, as for me, I could go for some of that. Those Iraqi chicks have got it going on.''

Borislav leaned on his elbows and ripped up a mouthful of bread. He poured another shot, downed it, then fell silent as the booze stole up on him in a rush. He was suddenly done with talk. Talk wasn't life. He'd seen real life. He knew it well. He'd first seen real life as a young boy, when he saw a whole population turned inside out. Refugees, the unemployed, the dispossessed, people starting over with pencils in a tin cup, scraping a living out of suitcases. Then people moving into stalls and kiosks.

``Transition,'' that's what they named that kind of life. As if it were all going somewhere in particular. The world changed a lot in a Transition. Life changed. But the people never transitioned into any rich nation's notion of normal life. In the next big ``transition,'' the twenty-first-century one, the people lost everything they had gained. When Borislav crutched back, maimed, from the outbreak of shooting-and-looting, he threw a mat on the sidewalk. He sold people boots. The people needed his boots, even indoors, because there was no more fuel in the pipelines and the people were freezing.

Come summer, he got hold of a car. Whenever there was diesel or bio-fuel around, he sold goods straight from its trunk. He made some street connections. He got himself a booth on the sidewalk. 

Even in the rich countries, the lights were out and roads were still. The sky was empty of jets. It was a hard Transition. Civilization was wounded. Then a contagion swept the world. Economic depression was bad, but a plague was a true Horseman of the Apocalypse. Plague thundered through a city. Plague made a city a place of thawing ooze, spontaneous fires, awesome deadly silences.

Borislav moved from his booth into the freezing wreck of a warehouse, where the survivors sorted and sold the effects of the dead. Another awful winter. They burned furniture to stay warm. When they coughed, people stared in terror at their handkerchiefs. Food shortages, too, this time: the dizzy edge of famine. Crazy times. He had nothing left of that former life but his pictures. During the mayhem, he took thousands of photographs. That was something to mark the day, to point a lens, to squeeze a button, when there was nothing else to do, except to hustle, or sit and grieve, or jump from a bridge. He still had all those pictures, every last one of them. Everyday photographs of extraordinary times. His own extraordinary self: he was young, gaunt, wounded, hungry, burning-eyed.

As long as a man could recognize his own society, then he could shape himself to fit its circumstances. He might be a decent man, dependable, a man of his word. But when the society itself was untenable, when it just could not be sustained--then ``normality'' cracked like a cheap plaster mask. Beneath the mask of civilization was another face: the face of a cannibal child. Only hope mattered then: the will to carry on through another day, another night, with the living strength of one's own heartbeat, without any regard for abstract notions of success or failure. In real life, to live was the only ``real.''

In the absence of routine, in the vivid presence of risk and suffering, the soul grew. Objects changed their primal nature. Their value grew as keen as tears, as keen as kisses. Hot water was a miracle. Electric light meant instant celebration. A pair of boots was the simple, immediate reason that your feet had not frozen and turned black. A man who had toilet paper, insulation, candles: he was the people's hero. When you handed a woman a tube of lipstick, her pinched and pallid face lit up all over. She could smear that scarlet on her lips, and when she walked down the darkened street it was as if she were shouting aloud, as if she were singing. When the plague burned out--it was a flu, and it was a killer, but it was not so deadly as the numb despair it inspired--then a profiteer's fortune beckoned to those tough enough to knock heads and give orders. Borislav made no such fortune. He knew very well how such fortunes were made, but he couldn't give the orders. He had taken orders himself, once. Those were orders he should never have obeyed. Like a stalled train, civilization slowly rattled back into motion, with its usual burden of claptrap. The life he had now, in the civilized moving train, it was a parody of that past life. That burning, immediate life. He had even been in love then. Today, he lived inside his kiosk. It was a pretty nice kiosk, today. Only a fool could fail to make a living in good times. He took care, he improvised, so he made a profit. He was slowly buying up some flats in an old apartment building, an ugly, unloved place, but sturdy and well-located. When old age stole over him, when he was too weak to hustle in the market anymore, then he would live on the rents. A football team scored on the big flat screen. The regulars cheered and banged their flimsy tables. Borislav raised his heavy head, and the bar's walls reeled as he came to himself. He was such a cheap drunk now; he would really have to watch that.

\bigskip\centerline{{$\bigtriangleup\bigtriangledown\bigtriangleup\bigtriangledown\bigtriangleup$}}\bigskip

Morning was painful. Borislav's mother tiptoed in with muesli, yogurt, and coffee. Borislav put his bad foot into his mother's plastic footbath--that treatment often seemed to help him--and he paged through a crumbling yellow block of antique newspapers. The old arts district had always been a bookish place, and these often showed up in attics. Borislav never read the ancient ``news'' in the newspapers, which, during any local regime, consisted mostly of official lies. Instead, he searched for the strange things that the people had once desired. Three huge, universal, dead phenomena haunted these flaking pages: petroleum cars, cinema, and cigarettes. The cars heavily dragged along their hundreds of objects and services: fuels, pistons, mufflers, makers of sparks. The cigarettes had garish paper packages, with lighters, humidors, and special trays just for their ashes. As for the movie stars, they were driving the cars and smoking the cigarettes.
The very oldest newspapers were downright phantasmagoric. All the newspapers, with their inky, frozen graphics, seemed to scream at him across their gulf of decades. The dead things harangued, they flattered, they shamed, they jostled each other on the paper pages. They bled margin-space, they wept ink. These things were strange, and yet, they had been desired. At first with a sense of daring, and then with a growing boldness, Borislav chose certain dead items to be digitally copied and revived. He re-released them into the contemporary flow of goods. For instance, by changing its materials and proportions, he'd managed to transform a Soviet-era desk telephone into a lightweight plastic rain-hat. No one had ever guessed the origin of his experiments. Unlike the machine-generated new products--always slotted with such unhuman coolness into market niches--these revived goods stank of raw humanity. Raw purpose.
Raw desire. Once, there had been no Internet. And no Internet-of-Things, either, for that could only follow. There had only been the people. People wanting things, trying to make other people want their things. Capitalism, socialism, communism, those mattered little enough. Those were all period arrangements in a time that had no Internet.

The day's quiet study restored Borislav's good spirits. Next morning, his mother recommenced her laments about her lack of a daughter-in-law. Borislav left for work.

He found his kiosk pitifully stripped and empty, with a Closed sign in its damp-spotted window. A raw hole loomed in the wall where the fabrikator had been torn free. This sudden loss of all his trade-goods gave him a lofty thrill of panic. Borislav savored that for a moment, then put the fear behind him. The neighborhood still surrounded his kiosk. The people would nourish it. He had picked an excellent location, during the darkest days. Once he'd sold them dirty bags of potatoes here, they'd clamored for wilted carrots. This life was easy now. This life was like a good joke.
He limped through the biometric door and turned on the lights. Now, standing inside, he felt the kiosk's true nature. A kiosk was a conduit. It was a temporary stall in the endless flow of goods. 
His kiosk was fiberboard and glue: recycled materials, green and modern. It had air filters, insulated windows, a rugged little fuel cell, efficient lights, a heater grill in the floor. It had password-protected intrusion alarms. It had a medical scanner in the walls. It had smart-ink wallpaper with peppy graphics.

They had taken away his custom-shaped chair, and his music player, loaded with a fantastic mashed-up mulch of the complete pop hits of the twentieth century. He would have to replace those. That wouldn't take him long. He knelt on the bare floor, and taped a thick sheet of salvaged cardboard over the wintry hole in his wall. A loud rap came at his window. It was Fleka the Gypsy, one of his suppliers. Borislav rose and stepped outside, reflexively locking his door, since this was Fleka. Fleka was the least dependable of his suppliers, because Fleka had no sense of time. Fleka could make, fetch, or filch most anything, but if you dared to depend on his word, Fleka would suddenly remember the wedding of some gypsy cousin, and vanish.

 ``I heard about your good luck, Boots,'' grinned Fleka. ``Is the maestro in need of new stock?'' 
 
Borislav rapped the empty window with his cane. ``It's as you see.''

Fleka slid to the trunk of his rusty car and opened it.

``Whatever that is,'' said Borislav at once, ``it's much too big.''

``Give me one minute from your precious schedule, maestro,'' said Fleka.

``You, my kind old friend, with your lovely kiosk so empty, I didn't bring you any goods. I brought you a factory! So improved! So new!''

``That thing's not new, whatever it is.''

``See, it's a fabrikator! Just like the last fabrikator I got for you, only this one is bigger, fancy and much better! I got it from my cousin.''

``I wasn't born yesterday, Fleka.''

Fleka hustled under his back seat and brought out a sample. It was a rotund doll of the American actress, Marilyn Monroe. The doll was still unpainted. It was glossily black.
Marilyn Monroe, the ultimate retail movie-star, was always recognizable, due to her waved coif, her lip-mole, and her torpedo-like bust. The passage of a century had scarcely damaged her shelf-appeal. The woman had become an immortal cartoon, like Betty Boop.

Fleka popped a hidden seam under Marilyn's jutting bust. Inside the black Marilyn doll was a smaller Marilyn doll, also jet-black, but wearing less clothing. Then came a smaller, more risqu\'e little Marilyn, and then a smaller one yet, and finally a crudely modelled little Marilyn, shiny black, nude, and the size of Borislav's thumb.

``Nice celebrity branding,'' Borislav admitted. ``So what's this material?'' It seemed to be black china.

``It's not wax, like that other fabrikator. This is carbon. Little straws of carbon. It came with the machine.''

Borislav ran his thumbnail across the grain of the material. The black Marilyn doll was fabricated in ridges, like the grooves in an ancient gramophone record. Fabs were always like that: they jet-sprayed their things by piling up thin layers, they stacked them up like pancakes. 

`` `Little straws of carbon.' I never heard of that.''

``I'm telling you what my cousin told me. `Little nano tubes, little nano carbon.' That's what he said.''

Fleka grabbed the round Marilyn doll like a football goalie, and raised both his hands overhead. Then, with all his wiry strength, he smacked the black doll against the rust-eaten roof of his car. Chips flew.

``You've ruined her!''

``That was my car breaking,'' Fleka pointed out. ``I made this doll this morning, out of old plans and scans from the Net. Then I gave it to my nephew, a nice big boy. I told him to break the doll. He broke a crowbar on this doll.''

Borislav took the black doll again, checked the seams and detailing, and rapped it with his cane. 

``You sell these dolls to anyone else, Fleka?''

``Not yet.''

``I could move a few of these. How much you asking?''

Fleka spread his hands. ``I can make more. But I don't know how to make the little straws of carbon. There's a tutorial inside the machine. But it's in Polish. I hate tutorials.''

Borislav examined the fabrikator. The machine looked simple enough: it was a basic black shell, a big black hopper, a black rotating plate, a black spraying nozzle, and the black gearing of a 3-D axis. ``Why is this thing so black?''

``It's nice and shiny, isn't it? The machine itself is made of little straws of carbon.''

``Your cousin got you this thing? Where's the brand name? Where's the serial number?''

``I swear he didn't steal it! This fabrikator is a copy, see. It's a pirate copy of another fabrikator in Warsaw. But nobody knows it's a copy. Or if they do know, the cops won't be looking for any copies around this town, that's for sure.''

Borislav's doubts overflowed into sarcasm. ``You're saying it's a fabrikator that copies fabrikators? It's a fabbing fab fabber, that's what you're telling me, Fleka?''

A shrill wail of shock and alarm came from the front of the kiosk. Borislav hurried to see.
A teenage girl, in a cheap red coat and yellow winter boots, was sobbing into her cellphone. She was Jovanica, one of his best customers.

``What's the matter?'' he said.

``Oh! It's you!'' Jovanica snapped her phone shut and raised a skinny hand to her lips. ``Are you still alive, Mr. Boots?''

``Why wouldn't I be alive?''

``Well, what happened to you? Who robbed your store?''

``I'm not robbed. Everything has been sold, that's all.''

Jovanica's young face screwed up in doubt, rage, frustration and grief. ``Then where are my hair toys?''

``What?''

``Where are my favorite barrettes? My hair clips! My scrunchies and headbands and beautiful pins! There was a whole tree of them, right here! I picked new toys from that tree every day! I finally had it giving me just what I wanted!''

``Oh. That.'' Borislav had sold the whirling rack of hair toys, along with its entire freight of goods. 

``Your rack sold the best hair toys in town! So super and cool! What happened to it? And what happened to your store? It's broken! There's nothing left!''

``That's true, `Neetsa. You had a very special relationship with that interactive rack, but\dots well\dots`` Borislav groped for excuses, and, with a leap of genius, he found one. ``I'll tell you a secret. You're growing up now, that's what.''

``I want my hair toys! Go get my rack right now!''

``Hair toys are for the nine-to-fifteen age bracket. You're growing out of that market niche. You should be thinking seriously about earrings.''

Jovanica's hands flew to her earlobes. ``You mean pierce my ears?''

Borislav nodded. ``High time.''

``Mama won't let me do that.''

``I can speak to your mama. You're getting to be a big girl now. Soon you'll have to beat the boys away with a stick.''

Jovanica stared at the cracks in the pavement. ``No I won't.''

``Yes you will,'' said Borislav, hefting his cane reflexively. Fleka the Gypsy had been an interested observer. Now he spoke up. ``Don't cry about your pretty things: because Boots here is the King of Kiosks. He can get you all the pretty things in the world!''

``Don't you listen to the gypsy,'' said Borislav. ``Listen, Jovanica: your old hair-toy tree, I'm sorry, it's gone for good. You'll have to start over with a brand new one. It won't know anything about what you want.''

``After all my shopping? That's terrible!''

``Never you mind. I'll make you a different deal. Since you're getting to be such a big girl, you're adding a lot of value by making so many highly informed consumer choices. So, next time, there will be a new economy for you. I'll pay you to teach that toy tree just what you want to buy.''
Fleka stared at him. ``What did you just say? You want to pay this kid for shopping?''

``That's right.''

``She's a little kid!''

``I'm not a little kid!'' Jovanica took swift offense. ``You're a dirty old gypsy!''

``Jovanica is the early hair-toy adopter, Fleka. She's the market leader here. Whatever hair toys Jovanica buys, all the other girls come and buy. So, yeah. I'm gonna cut her in on that action. I should have done that long ago.''

Jovanica clapped her hands. ``Can I have lots of extra hair toys, instead of just stupid money?''
``Absolutely. Of course. Those loyal-customer rewards will keep you coming back here, when you ought to be doing your homework.''

Fleka marveled. ``It's completely gone to your head, cashing out your whole stock at once. A man of your age, too.''

The arts district never lacked for busybodies. Attracted by the little drama, four of them gathered round Borislav's kiosk. When they caught him glowering at them, they all pretended to need water from his fountain. At least his fountain was still working.

``Here comes my mama,'' said Jovanica. Her mother, Ivana, burst headlong from the battered doors of a nearby block of flats. Ivana wore a belted house robe, a flung-on muffler, a heavy scarf, and brightly knitted woolen house slippers. She brandished a laden pillowcase.

``Thank God they haven't hurt you!'' said Ivana, her breath puffing in the chilly air. She opened her pillowcase. It held a steam iron, a hair dryer, an old gilt mirror, a nickeled hip-flask, a ragged fur stole, and a lidded, copper-bottomed saucepan.

``Mr. Boots is all right, Mama,'' said Jovanica. ``They didn't steal anything. He sold everything!''

``You sold your kiosk?'' said Ivana, and the hurt and shock deepened in her eyes. ``You're leaving us?''

``It was business,'' Borislav muttered. ``Sorry for the inconvenience. It'll be a while before things settle down.''

``Honestly, I don't need these things. If these things will help you in any way, you're very welcome to them.''

``Mama wants you to sell these things,'' Jovanica offered, with a teen's oppressive helpfulness. ``Then you can have the money to fix your store.''

Borislav awkwardly patted the kiosk's fiberboard wall. ``Ivana, this old place doesn't look like much, so empty and with this big hole ... but, well, I had some luck.''

``Ma'am, you must be cold in those house slippers,'' said Fleka the Gypsy. With an elegant swoop of his arm, he gestured at the gilt-and-glassed front counter of the Three Cats Caf. ``May I get you a hot cappuccino?''

``You're right, sir, it's cold here.'' Ivana tucked the neck of her pillowcase, awkwardly, over her arm.

``I'm glad things worked out for you, Borislav.''

``Yes, things are all right. Really.''

Ivana aimed a scowl at the passersby, who watched her with a lasting interest. ``We'll be going now, `Neetsa.''

``Mama, I'm not cold. The weather's clearing up!''

``We're going.'' 

They left.

Fleka picked at his discolored canine with his forefinger. ``So, maestro. What just happened there?''

``She's a nice kid. She's hasty sometimes. The young are like that. That can't be helped.'' Borislav shrugged. ``Let's talk our business inside.''

He limped into his empty kiosk. Fleka wedged in behind him and managed to slam the door. Borislav could smell the man's rich, goulash-tinged breath.

``I was never inside one of these before,'' Fleka remarked, studying every naked seam for the possible point of a burglar's pry-bar. ``I thought about getting a kiosk of my own, but, well, a man gets so restless.'' 

``It's all about the product flow divided by the floor space. By that measure, a kiosk is super-efficient retailing. It's about as efficient as any sole proprietor can do. But it's a one-man enterprise. So, well, a man's just got to go it alone.''

Fleka looked at him with wise, round eyes. ``That girl who cried so much about her hair. That's not your girl, is she?''

``What? No.''

``What happened to the father, then? The flu got him?''

``She was born long after the flu, but, yeah, you're right, her father passed away.'' Borislav coughed. ``He was a good friend of mine. A soldier. Really good-looking guy. His kid is gorgeous.''

``So you didn't do anything about that. Because you're not a soldier, and you're not rich, and you're not gorgeous.''

``Do anything about what?''

``A woman like that Ivana, she isn't asking for some handsome soldier or some rich-guy boss. A woman like her, she wants maybe a pretty dress. Maybe a dab of perfume. And something in her bed that's better than a hot-water bottle.''

``Well, I've got a kiosk and a broken leg.''

``All us men have a broken leg. She thought you had nothing. She ran right down here, with anything she could grab for you, stuffed into her pillowcase. So you're not an ugly man. You're a stupid man.'' Fleka thumped his chest. ``I'm the ugly man. Me. I've got three wives: the one in Bucharest, the one in Lublin, and the wife in Linz isn't even a gypsy. They're gonna bury me standing, maestro. That can't be helped, because I'm a man. But that's not what you are. You're a fool.''

``Thanks for the free fortune-telling. You know all about this, do you? She and I were here during the hard times. That's what. She and I have a history.''

``You're a fanatic. You're a geek. I can see through you like the windows of this kiosk. You should get a life.'' Fleka thumped the kiosk's wallpaper, and sighed aloud. ``Look, life is sad, all right? Life is sad even when you do get a life. So. Boots. Now I'm gonna tell you about this fabrikator of mine, because you got some spare money, and you're gonna buy it from me. It's a nice machine. Very sweet. It comes from a hospital. It's supposed to make bones. So the tutorial is all about making bones, and that's bad, because nobody buys bones. If you are deaf and you want some new little black bones in your ears, that's what this machine is for. Also, these black toys I made with it, I can't paint them. The toys are much too hard, so the paint breaks right off. Whatever you make with this fabrikator, it's hard and black, and you can't paint it, and it belongs by rights inside some sick person. Also, I can't read the stupid tutorials. I hate tutorials. I hate reading.''

``Does it run on standard voltage?''

``I got it running on DC off the fuel cell in my car.''

``Where's the feedstock?''

``It comes in big bags. It's a powder, it's a yellow dust. The fab sticks it together somehow, with sparks or something, it turns the powder shiny black and it knits it up real fast. That part, I don't get.''

``I'll be offering one price for your machine and all your feedstock.''

``There's another thing. That time when I went to Vienna. I gave you my word on that deal. We shook hands on it. That deal was really important, they really needed it, they weren't kidding about it, and, well, I screwed up. Because of Vienna.''

``That's right, Fleka. You screwed up bad.''

``Well, that's my price. That's part of my price. I'm gonna sell you this toy-maker. We're gonna haul it right out of the car, put it in the kiosk here nice and safe. When I get the chance, I'm gonna bring your bag  of coal-straw, too. But we forget about Vienna. We just forget about it.''

Borislav said nothing.

``You're gonna forgive me my bad, screwed-up past. That's what I want from you.''

``I'm thinking about it.''

``That's part of the deal.''

``We're going to forget the past, and you're going to give me the machine, the stock, and also fifty bucks.''

``Okay, sold.''

\bigskip\centerline{{$\bigtriangleup\bigtriangledown\bigtriangleup\bigtriangledown\bigtriangleup$}}\bigskip

With the fabrikator inside his kiosk, Borislav had no room inside the kiosk for himself. He managed to transfer the tutorials out of the black, silent fab and into his laptop. The sun had come out. Though it was still damp and chilly, the boys from the Three Cats had unstacked their white caf\'e chairs. Borislav took a seat there. He ordered black coffee and began perusing awkward machine translations from the Polish manual.

Selma arrived to bother him. Selma was married to a schoolteacher, a nice guy with a steady job.
Selma called herself an artist, made jewelry, and dressed like a lunatic. The schoolteacher thought the world of Selma, although she slept around on him and never cooked him a decent meal.

``Why is your kiosk so empty? What are you doing, just sitting out here?''

Borislav adjusted the angle of his screen. ``I'm seizing the means of production.''

``What did you do with all my bracelets and necklaces?''

``I sold them.''

``All of them?''

``Every last scrap.''

Selma sat down as it hit with a mallet. ``Then you should buy me a glass of champagne!''

Borislav reluctantly pulled his phone and text-messaged the waiter. It was getting blustery, but Selma preened over her glass of cheap Italian red.

``Don't expect me to replace your stock soon! My artwork's in great demand.''

``There's no hurry.''

``I broke the luxury market, across the river at the Intercontinental! The hotel store will take all the bone-ivory chokers I can make.''

``Mmm-hmm.''

``Bone-ivory chokers, they're the perennial favorite of ugly, aging tourist women with wattled necks.''

Borislav glanced up from his screen. ``Shouldn't you be running along to your workbench?''

``Oh, sure, sure, `give the people what they want,' that's your sick, petit-bourgeois philosophy! Those foreign tourist women in their big hotels, they want me to make legacy kitsch!''

Borislav waved one hand at the street. ``Well, we do live in the old arts district.''

``Listen, stupid, when this place was the young arts district, it was full of avant-gardists plotting revolution. Look at me for once. Am I from the museum?''

Selma yanked her skirt to mid-thigh. ``Do I wear little old peasant shoes that turn up at the toes?''

``What the hell has gotten into you? Did you sit on your tack-hammer?''

Selma narrowed her kohl-lined eyes. ``What do you expect me to do, with my hands and my artisan skills, when you're making all kinds of adornments with fabrikators? I just saw that stupid thing inside your kiosk there.''

Borislav sighed. ``Look, I don't know. You tell me what it means, Selma.''

``It means revolution. That's what. It means another revolution.''

Borislav laughed at her.

Selma scowled and lifted her kid-gloved fingers. ``Listen to me. Transition number one. When communism collapsed. The people took to the streets. Everything privatized. There were big market shocks.''

``I remember those days. I was a kid, and you weren't even born then.''

``Transition Two. When globalism collapsed. There was no oil. There was war and bankruptcy. There was sickness. That was when I was a kid.''

Borislav said nothing about that. All things considered, his own first Transition had been a kinder time to grow up in.

``Then comes Transition Three.'' Selma drew a breath. ``When this steadily increasing cybernetic intervention in manufacturing liberates a distinctly human creativity.''

``Okay, what is that about?''

``I'm telling you what it's about. You're not listening. We're in the third great Transition. It's  a revolution. Right now. Here. This isn't Communism, this isn't Globalism. This is the next thing after that. It's happening. No longer merely reacting to this influx of mindless goods, the modern artist uses human creative strength in the name of a revolutionary heterogeneity!''

Selma always talked pretentious, self-important drivel. Not quite like this, though. She'd found herself some new drivel.

``Where did you hear all that?''

``I heard it here in this caf\'e! You're just not listening, that's your problem. You never listen to anybody. Word gets around fast in the arts community.''

``I live here too, you know. I'd listen to your nutty blither all day, if you ever meant business.''
Selma emptied her wineglass. Then she reached inside her hand-loomed, artsy sweater. ``If you laugh at this, I'm going to kill you.''

Borislav took the necklace she offered him. ``Where's this from? Who sent you this?''

``That's mine! I made it. With my hands.''

Borislav tugged the tangled chain through his fingers. He was no jeweller, but knew what decent jewelry looked like. This was indecent jewelry. If the weirdest efforts of search engines looked like products from Mars, then this necklace was straight from Venus. It was slivers of pot-metal, blobs of silver, and chips of topaz. It was like jewelry straight out of a nightmare.

``Selma, this isn't your customary work.''

``Machines can't dream. I saw this in my dreams.''

``Oh. Right, of course.''

``Well, it was my nightmare, really. But I woke up! Then I created my vision! I don't have to make that cheap, conventional crap, you know! I only make cheap junk because that's all you are willing to sell!''

``Well\dots'' He had never spoken with frankness to Selma before, but the glittering light in her damp eyes made yesterday's habits seem a little slow-witted. ``Well, I wouldn't know what to charge for a work of art like this.''

``Somebody would want this, though? Right? Wouldn't they?'' She was pleading with him. ``Somebody? They would buy my new necklace, right? Even though it's ... different.''

``No. This isn't the sort of jewelry that the people buy. This is the sort of jewelry that the people stare at, and probably laugh at, too. But then, there would come one special person. She would really want this necklace. She would want this more than anything. She would have to have this thing at absolutely any price.''

``I could make more like that,'' Selma told him, and she touched her heart. ``Because now I know where it comes from.''

\vfil
\centerline{{$\bigtriangleup\bigtriangledown\bigtriangleup\bigtriangledown\bigtriangleup$}}

\chapter*{\centerline{III}}
Borislav installed the fab inside the empty kiosk, perched on a stout wooden pedestal, where its workings could be seen by the people. His first choices for production were, naturally, hair toys. Borislav borrowed some fancy clips from Jovanica, and copied their shapes inside his kiosk with his medical scanner.

Sure enough, the fabricator sprayed out shiny black replicas. Jovanica amused a small crowd by jumping up and down on them. The black clips themselves were well-nigh indestructible, but their cheap metal springs soon snapped.

Whenever a toy broke, however, it was a simple matter to cast it right back into the fabrikator's hopper. The fab chewed away at the black object, with an ozone-like reek, until the fabbed object became the yellow dust again. Straw, right into gold.

Borislav sketched out a quick business plan on the back of a Three Cats beer coaster. With hours of his labor, multiplied by price-per-gram, he soon established his point of profit. He was in a new line of work.

With the new fabrikator, he could copy the shapes of any small object he could scan. Of course, he couldn't literally ``copy'' everything: a puppy dog, a nice silk dress, a cold bottle of beer, those were all totally out of the question. But he could copy most anything that was made from some single, rigid material: an empty bottle, a fork, a trash can, a kitchen knife.

The kitchen knives were an immediate hit. The knives were shiny and black, very threatening and scary, and it was clear they would never need sharpening. It was also delightful to see the fabrikator mindlessly spitting up razor-sharp knives. The kids were back in force to watch the action, and this time, even the grownups gathered and chattered.

To accommodate the eager crowd of gawkers, the Three Cats boys set out their chairs and tables, and even their striped, overhead canopy, as cheery as if it were summer.

The weather favored them. An impromptu block party broke out. Mirko from the Three Cats gave him a free meal. ``I'm doing very well by this,'' Mirko said. ``You've got yourself a nine-days wonder here. This sure reminds me of when Transition Two was ending. Remember when those city lights came back on? Brother, those were great days.''

``Nine days won't last long. I need to get back inside that kiosk, like normal again.''

``It's great to see you out and about, mixing it up with us, Boots. We never talk anymore.'' Mirko spread his hands in apology, then scrubbed the table. ``I run this place now\dots it's the pressure of business\dots that's all my fault.''

Borislav accepted a payment from a kid who'd made himself a rock-solid black model dinosaur.
``Mirko, do you have room for a big vending machine, here by your caf\'e? I need to get that black beast out of my kiosk. The people need their sticks of gum.''

``You really want to build some vendorizing thing out here? Like a bank machine?''

``I guess I do, yeah. It pays.''

``Boots, I love this crowd you're bringing me, but why don't you just put your machine wherever they put bank machines? There are hundreds of bank machines.'' Mirko took his empty plate. ``There are millions of bank machines. Those machines took over the world.''

\vfil 
\centerline{{$\bigtriangleup\bigtriangledown\bigtriangleup\bigtriangledown\bigtriangleup$}}

\chapter*{\centerline{IV}}

Days passed. The people wouldn't let him get back to normal. It became a public sport to see what people would bring in for the fabrikator to copy. It was common to make weird things as gag gifts: a black, rock-solid spray of roses, for instance. You could hand that black bouquet to your girlfriend for a giggle, and if she got huffy, then you could just bring it back, have it weighed, and get a return-deposit for the yellow dust.

The ongoing street drama was a tonic for the neighborhood. In no time flat, every caf\'e lounger and class-skipping college student was a self-appointed expert about fabs, fabbing, and revolutionary super-fabs that could fab their own fabbing. People brought their relatives to see. Tourists wandered in and took pictures. Naturally they all seemed to want a word with the owner and proprietor. The people being the people, the holiday air was mixed with unease. Things took a strange turn when a young bride arrived with her wedding china. She paid to copy each piece, then loudly and publicly smashed the originals in the street. A cop showed up to dissuade her. Then the cop wanted a word, too. Borislav was sitting with Professor Damov, an academician and pious blowhard who ran the local ethnographic museum. The professor's city-sponsored hall specialized in what Damov called ``material culture,'' meaning dusty vitrines full of battle flags, holy medallions, distaffs, fishing nets, spinning wheels, gramophones, and such. Given these new circumstances, the professor had a lot on his mind.

``Officer,'' said Damov, briskly waving his wineglass, ``it may well surprise you to learn this, but the word `kiosk' is an ancient Ottoman term. In the original Ottoman kiosk, nothing was bought or sold. The kiosk was a regal gift from a prince to the people. A kiosk was a place to breathe the evening air, to meditate, to savor life and living; it was an elegant garden pergola.''

``They didn't break their wedding china in the gutter, though,'' said the cop.

``Oh, no, on the contrary, if a bride misbehaved in those days, she'd be sewn into a leather sack and thrown into the Bosphorus!''

The cop was mollified, and he moved right along, but soon a plainclothes cop showed up and took a prominent seat inside the Three Cats Caf\'e. This changed the tone of things. The police surveillance proved that something real was happening. It was a kind of salute.

Dusk fell. A group of garage mechanics came by, still in their grimy overalls, and commenced a deadly serious professional discussion about fabbing trolley parts. A famous stage actor showed up with his entourage, to sign autographs and order drinks for all his ``friends.''

Some alarmingly clean-cut university students appeared. They weren't there to binge on beer. They took a table, ordered Mirko's cheapest pizza, and started talking in points-of-order.

Next day, the actor brought the whole cast of his play, and the student radicals were back in force. They took more tables, with much more pizza. Now they had a secretary, and a treasurer. Their ringleaders had shiny black political buttons on their coats.

A country bus arrived and disgorged a group of farmers. These peasants made identical copies of something they were desperate to have, yet anxious to hide from all observers.

Ace came by the bustling caf\'e. Ace was annoyed to find that he had to wait his turn for any private word with Borislav.

``Calm down, Ace. Have a slice of this pork pizza. The boss here's an old friend of mine, and he's in a generous mood.''

``Well, my boss is unhappy,'' Ace retorted. ``There's money being made here, and he wasn't told about it.''

``Tell your big guy to relax. I'm not making any more money than I usually do at the kiosk. That should be obvious: consider my rate of production. That machine can only make a few copies an hour.''

``Have you finally gone stupid? Look at this crowd!'' Ace pulled his shades off and studied the densely clustered caf\'e. Despite the lingering chill, a gypsy band was setting up, with accordions and trombones.

``Okay, this proves it: see that wise guy sitting there with that undercover lieutenant? He's one of them!''

Borislav cast a sidelong glance at the rival gangster. The North River Boy looked basically identical to Ace: the same woolly hat, cheap black sunglasses, jacket, and bad attitude, except for his sneakers, which were red instead of blue.

``The River Boys are moving in over here?''

``They always wanted this turf. This is the lively part of town.''

That River Boy had some nerve. Gangsters had been shot in the Three Cats Caf\'e. And not just a few times, either. It was a major local tradition.

``I'm itching to whack that guy,'' Ace lied, sweating, ``but, well, he's sitting over there with that cop! And a pet politician, too!''

Borislav wondered if his eyes were failing. In older days, he would never have missed those details.

There was a whole little tribe of politicians filtering into the caf\'e, and sitting near the mobster's table.

The local politicians always traveled in parties. Small, fractious parties.

One of these local politicals was the arts districts' own national representative. Mr. Savic was a member of the Radical Liberal Democratic Party, a splinter clique of well-meaning, overeducated cranks. ``I'm gonna tell you a good joke, Ace. `You can get three basic qualities with any politician: Smart, Honest, and Effective. But you only get to pick two.' ''

Ace blinked. He didn't get it.

Borislav levered himself from his caf\'e chair and limped over to provoke a gladhanding from Mr. Savic.

The young lawyer was smart and honest, and therefore ineffective. However, Savic, being so smart, was quick to recognize political developments within his own district. He had already appropriated the shiny black button of the young student radicals.

With an ostentatious swoop of his camel's-hair coattails, Mr. Savic deigned to sit at Borislav's table.

He gave Ace a chilly glare. ``Is it necessary that we consort with this organized-crime figure?''

``You tried to get me fired from my job in the embassy,'' Ace accused him.

``Yes, I did. It's bad enough that the criminal underworld infests our ruling party. We can't have the Europeans paying you off, too.''

``That's you all over, Savic: always sucking up to rich foreigners and selling out the guy on the street!''

``Don't flatter yourself, you jumped-up little crook! You're not `the street.' The people are the street!''

``Okay, so you got the people to elect you. You took office and you got a pretty haircut. Now you're gonna wrap yourself up in our flag, too? You're gonna steal the last thing the people have left!''

Borislav cleared his throat. ``I'm glad we have this chance for a frank talk here. The way I figure it, managing this fabbing business is going to take some smarts and finesse.''

The two of them stared at him. ``You brought us here?'' Ace said. ``For our `smarts and finesse?'''
``Of course I did. You two aren't here by accident, and neither am I. If we're not pulling the strings around here, then who is?''

The politician looked at the gangster. ``There's something to what he says, you know. After all, this is Transition Three.''

``So,'' said Borislav, ``knock it off with that tough talk and do some fresh thinking for once! You sound like your own grandfathers!''

Borislav had surprised himself with this outburst. Savic, to his credit, looked embarrassed, while Ace scratched uneasily under his woolly hat. ``Well, listen, Boots,'' said Ace at last, ``even if you, and me, and your posh lawyer pal have us three nice Transition beers together, that's a River Boy sitting over there. What are we supposed to do about that?''

``I am entirely aware of the criminal North River Syndicate,'' Mr. Savic told him airily. ``My investigative committee has been analyzing their gang.''

``Oh, so you're analyzing, are you? They must be scared to death.''

``There are racketeering laws on the books in this country,'' said Savic, glowering at Ace. ``When we take power and finally have our purge of the criminal elements in this society, we won't stop at arresting that one little punk in his cheap red shoes. We will liquidate his entire parasite class: I mean him, his nightclub-singer girlfriend, his father, his boss, his brothers, his cousins, his entire football club\dots As long as there is one honest judge in this country, and there are some honest judges, there are always some\dots We will never rest! Never!''

``I've heard about your honest judges,'' Ace sneered. ``You can spot `em by the smoke columns when their cars blow up.''

``Ace, stop talking through your hat. Let me make it crystal clear what's at stake here.'' Borislav reached under the table and brought up a clear plastic shopping bag. He dropped it on the table with a thud.

Ace took immediate interest. ``You output a skull?''

``Ace, this is my own skull.'' The kiosk scanned him every day. So Borislav had his skull on file.
Ace juggled Borislav's skull free of the clear plastic bag, then passed it right over to the politician.

``That fab is just superb! Look at the crisp detailing on those sutures!''

``I concur. A remarkable technical achievement.'' Mr. Savic turned the skull upside down, and frowned.

``What happened to your teeth?''

``Those are normal.''

``You call these wisdom teeth normal?''

``Hey, let me see those,'' Ace pleaded. Mr. Savic rolled Borislav's jet-black skull across the tabletop.

Then he cast an over-shoulder look at his fellow politicians, annoyed that they enjoyed themselves so much without him.

``Listen to me, Mr. Savic. When you campaigned, I put your poster up in my kiosk. I even voted for you, and--''

Ace glanced up from the skull's hollow eye-sockets. ``You vote, Boots?''

``Yes. I'm an old guy. Us old guys vote.''

Savic faked some polite attention.

``Mr. Savic, you're our political leader. You're a Radical Liberal Democrat. Well, we've got ourselves a pretty radical, liberal situation here. What are we supposed to do now?''

``It's very good that you asked me that,'' nodded Savic. ``You must be aware that there are considerable intellectual-property difficulties with your machine.''

``What are those?''

``I mean patents and copyrights. Reverse-engineering laws. Trademarks. We don't observe all of those laws in this country of ours ... in point of fact, practically speaking, we scarcely observe any\dots But the rest of the world fully depends on those regulatory structures. So if you go around publicly pirating wedding china--let's just say--well, the makers of wedding china will surely get wind of that some day. I'd be guessing that you see a civil lawsuit. Cease-and-desist, all of that.''

``I see.''

``That's just how the world works. If you damage their income, they'll simply have to sue you. Follow the money, follow the lawsuits. A simple principle, really. Although you've got a very nice little sideshow here\dots It's really brightened up the neighborhood\dots''

Professor Damov arrived at the caf\'e. He had brought his wife, Mrs. Professor-Doctor Damova, an icy sociologist with annoying Marxist and feminist tendencies. The lady professor wore a fur coat as solid as a bank vault, and a bristling fur hat.

Damov pointed out a black plaque on Borislav's tabletop. ``I'm sorry, gentleman, but this table is reserved for us.''

``Oh,'' Borislav blurted. He hadn't noticed the fabbed reservation, since it was so black.

``We're having a little party tonight,'' said Damov, ``it's our anniversary.''

``Congratulations, sir and madame!'' said Mr. Savic. ``Why not sit here with us just a moment until your guests arrive?''

A bottle of Mirko's prosecco restored general good feeling. ``I'm an arts-district lawyer, after all,'' said Savic, suavely topping up everyone's glasses.

``So, Borislav, if I were you, I would call this fabrikator an arts installation!''

``Really? Why?''

``Because when those humorless foreigners with their lawsuits try to make a scandal of the arts scene, that never works!'' Savic winked at the professor and his wife. ``We really enjoyed it, eh? We enjoyed a good show while we had it!''

Ace whipped off his sunglasses. ``It's an `arts installation!' Wow! That is some smart lawyer thinking there!''

Borislav frowned. ``Why do you say that?''

Ace leaned in to whisper behind his hand. ``Well, because that's what we tell the River Boys! We tell them it's just an art show, then we shut it down. They stay in their old industrial district, and we keep our turf in the old arts district. Everything is cool!''

``That's your big solution?''

``Well, yeah,'' said Ace, leaning back with a grin. ``Hooray for art!''

Borislav's temper rose from a deep well to burn the back of his neck. ``That's it, huh? That's what you two sorry sons of bitches have to offer the people? You just want to get rid of the thing! You want to put it out, like spitting on a candle! Nothing happens with your stupid approach! You call that a Transition? Everything's just the same as it was before! Nothing changes at all!''

Damov shook his head. ``History is always passing. We changed. We're all a year older.''

Mrs. Damov spoke up. ``I can't believe your fascist, technocratic nonsense! Do you really imagine that you will improve the lives of the people by dropping some weird machine onto their street at random? With no mature consideration of any deeper social issues? I wanted to pick up some milk tonight! Who's manning your kiosk, you goldbricker? Your store is completely empty! Are we supposed to queue?''

Mr. Savic emptied his glass. ``Your fabrikator is great fun, but piracy is illegal and immoral. Fair is fair, let's face it.''

``Fine,'' said Borislav, waving his arms, ``if that's what you believe, then go tell the people. Tell the people in this caf\'e, right now, that you want to throw the future away! Go on, do it! Say you're scared of crime! Say they're not mature enough and they have to think it through. Tell the people that they have to vote for that!''

``Let's not be hasty,'' said Savic.

``Your sordid mechanical invention is useless without a social invention,'' said Mrs. Damova primly.

``My wife is exactly correct!'' Damov beamed. ``Because a social invention is much more than gears and circuits, it's ... well, it's something like that kiosk. A kiosk was once a way to drink tea in a royal garden. Now it's a way to buy milk! That is social invention!'' He clicked her bubbling glass with his own. 

Ace mulled this over. ``I never thought of it that way. Where can we steal a social invention? How do you copy one of those?''

These were exciting questions. Borislav felt a piercing ray of mental daylight.

``That European woman, what's-her-face. She bought out my kiosk. Who is she? Who does she work for?''

``You mean Dr. Grootjans? She is, uh \dots she's the economic affairs liaison for a European Parliamentary investigative committee.''

``Right,'' said Borislav at once, ``that's it. Me, too! I want that. Copy me that! I'm the liaison for the investigation Parliament something stupid-or-other.''

Savic laughed in delight. ``This is getting good.''

``You. Mr. Savic. You have a Parliament investigation committee.''

``Well, yes, I certainly do.''

``Then you should investigate this fabrikator. You place it under formal government investigation. You investigate it, all day and all night. Right here on the street, in public. You issue public reports. And of course you make stuff. You make all kinds of stuff. Stuff to investigate.''

``Do I have your proposal clear? You are offering your fabrikator to the government?''

``Sure. Why not? That's better than losing it. I can't sell it to you. I've got no papers for it. So sure, you can look after it. That's my gift to the people.''

Savic stroked his chin. ``This could become quite an international issue.''

Suddenly, Savic had the look of a hungry man about to sit at a bonfire and cook up a whole lot of sausages.

"Man, that's even better than making it a stupid art project,'' Ace enthused.

"A stupid government project! Hey, those last forever!''

\vfil
\centerline{{$\bigtriangleup\bigtriangledown\bigtriangleup\bigtriangledown\bigtriangleup$}}

\chapter*{\centerline{V}}

Savic's new investigation committee was an immediate success. With the political judo typical of the region, the honest politician wangled a large and generous support grant from the Europeans--basically, in order to investigate himself. The fab now reformed its efforts: from consumer knickknacks to the pressing needs of the state's public sector. Jet-black fire plugs appeared in the arts district. Jet-black hoods for the broken street lights, and jet-black manhole covers for the streets. Governments bought in bulk, so a primary source for the yellow dust was located. The fab churned busily away right in the public square, next to a railroad tanker full of feed-stock.

Borislav returned to his kiosk. He made a play at resuming his normal business. He was frequently called to testify in front of Savic's busy committee. This resulted in Fleka the Gypsy being briefly arrested, but the man skipped bail. No one made any particular effort to find Fleka. They certainly had never made much effort before.

Investigation soon showed that the fabrikator was stolen property from a hospital in Gdansk. Europeans had long known how to make such fabrikators: fabrikators that used carbon nanotubes. They had simply refrained from doing so. As a matter of wise precaution, the Europeans had decided not to create devices that could so radically disrupt a well-established political and economic order. The pain of such an act was certain to be great. The benefits were doubtful. On some grand, abstract level of poetic engineering, it obviously made sense to create super-efficient, widely distributed, cottage-scale factories that could create as much as possible with as little as possible. If one were inventing industrial civilization from the ground up, then fabbing was a grand idea. But an argument of that kind made no sense to the installed base and the established interests. You couldn't argue a voter out of his job. So fabs had been subtly restricted to waxes, plastics, plaster, papier-m\^ach\'e, and certain metals.

Except, that is, for fabs with medical applications. Medicine, which dealt in agonies of life and death, was never merely a marketplace. There was always somebody whose child had smashed and shattered bones. Sooner or later this violently interested party, researching a cure for his beloved, would find the logjam and scream: won't one of you heartless, inhuman bastards think of the children?

Of course, those who had relinquished this technology had the children's best interests at heart. They wanted their children to grow up safe within stable, regulated societies. But one could never explain good things for vaporous, potential future children to someone whose heart and soul was twisted by the suffering of an actual, real-life child.

So a better and different kind of fab had come into being. It was watched over with care ... but, as time and circumstance passed, it slipped loose. Eager to spread the fabbing pork through his constituency, Savic commissioned renowned local artists to design a new breed of kiosk. This futuristic Transition Three ultra-kiosk would house the very fab that could make it. Working with surprising eagerness and speed (given that they were on government salary), these artisan-designers created a new, official, state-supported fabbing kiosk, an alarmingly splendid, well-nigh monumental kiosk, half Ottoman pavilion, half Stalinist gingerbread, and almost one hundred percent black carbon nanotubes, except for a few necessary steel bolts, copper wires, and brass staples. Borislav knew better than to complain about this. He had to abandon his perfectly decent, old-fashioned, customary kiosk, which was swiftly junked and ripped into tiny recyclable shreds. Then he climbed, with pain and resignation, up the shiny black stair-steps into this eerie, oversized, grandiose rock-solid black fort, this black-panelled royal closet whose ornate, computer-calligraphic roof would make meteors bounce off it like graupel hail.

The cheap glass windows fit badly. The new black shelves confused his fingers. The slick black floor sent his chair skidding wildly. The black carbon walls would not take paint, glue or paper. He felt like an utter fool--but this kiosk hadn't been built for his convenience. This was a kiosk for the new Transition Three generation, crazily radical, liberal guys for whom a ``kiosk'' was no mere humble conduit, but the fortress of a new culture war.

A kiosk like this new one could be flung from a passing jet. It could hammer the ground like a plummeting thunderbolt and bounce up completely unharmed. With its ever-brimming bags of gold dust, a cybernetic tumbling of possessions would boil right out of it: bottles bags knobs latches wheels pumps, molds for making other things, tools for making other things; saws, hammers wrenches levers, drill-bits, screws, screwdrivers, awls, pliers, scissors, punches, planes, files, rasps, jacks, carts and shears; pulleys, chains and chain hoists, trolleys, cranes, buckets, bottles, barrels\dots All of these items sitting within their digital files as neat as chess pieces, sitting there like the very idea of chess pieces, like a mental chess set awaiting human desire to leap into being and action. 

As Borislav limped, each night, from his black battleship super-kiosk back to his mother's apartment, he could see Transition Three insinuating itself into the fabric of his city.
Transition One had once a look all its own: old socialist buildings of bad brick and substandard plaster, peeling like a secret leprosy, then exploding with the plastic branding symbols of the triumphant West: candy bars, franchised fried food, provocative lingerie.

Transition Two was a tougher business: he remembered it mostly for its lacks and privations. Empty stores, empty roads, crowds of bicycles, the angry hum of newfangled fuel cells, the cheap glitter of solar roofing, insulation stuffed everywhere like the paper in a pauper's shoes. Crunchy, mulchy-looking new construction. Grass on the rooftops, grass in the trolley-ways. Networking masts and dishes. Those clean, cold, flat-panel lights.

 This third Transition had its own native look, too. It was the same song and another verse. It was black.
 
It was jet-black, smooth, anonymous, shiny, stainless, with an occasional rainbow shimmer off the layers and grooves whenever the light was just right, like the ghosts of long-vanished oil slicks. Revolution was coming. The people wanted more of this game than the regime was allowing them to have. There were five of the fabs running in the city now. Because of growing foreign pressure against ``the dangerous proliferation,'' the local government wouldn't make any more fabrikators. So the people were being denied the full scope of their desire to live differently. The people were already feeling different inside, so they were going to take it to the streets. The politicians were feebly trying to split differences between ways of life that just could not be split. Did the laws of commerce exist for the people's sake? Or did the people exist as slaves of the so-called laws of commerce? That was populist demagoguery, but that kind of talk was popular for a reason.

Borislav knew that civilization existed through its laws. Humanity suffered and starved whenever outside the law. But those stark facts didn't weigh on the souls of the locals for ten seconds. The local people here were not that kind of people. They had never been that kind of people. Turmoil: that was what the people here had to offer the rest of the world.

The people had flown off the handle for far less than this; for a shot fired at some passing prince, for instance. Little street demonstrations were boiling up from left and right. Those demonstrations waxed and waned, but soon the applecart would tip hard. The people would take to the city squares, banging their jet-black kitchen pans, shaking their jet-black house keys. Borislav knew from experience that this voice from the people was a nation-shaking racket. The voice of reason from the fragile government sounded like a cartoon mouse.

Borislav looked after certain matters, for there would be no time to look after them, later. He talked to a lawyer and made a new will. He made backups of his data and copies of important documents, and stashed things away in numerous caches. He hoarded canned goods, candles, medicines, tools, even boots. He kept his travel bag packed.

He bought his mother her long-promised cemetery plot. He acquired a handsome headstone for her, too.

He even found silk sheets.

\vfil
\centerline{{$\bigtriangleup\bigtriangledown\bigtriangleup\bigtriangledown\bigtriangleup$}}

\chapter*{\centerline{VI}}

It didn't break in the way he had expected, but then local history could be defined as events that no rational man would expect. It came as a kiosk. It was a brand new European kiosk. A civilized, ultimate, decent, well-considered, preemptive intervention kiosk. The alien pink-and-white kiosk was beautiful and perfect and clean, and there was no one remotely human inside it. The automatic kiosk had a kind of silver claw that unerringly picked its goods from its antiseptic shelves, and delivered them to the amazed and trembling customer. These were brilliant goods, they were shiny and gorgeous and tagged with serial numbers and radio-tracking stickers. They glowed all over with reassuring legality: health regulations, total lists of contents, cross-border shipping, tax stamps, associated websites, places to register a complaint.

The superpower kiosk was a thing of interlocking directorates, of 100,000-page regulatory codes and vast, clotted databases, a thing of true brilliance, neurosis and fine etiquette, like a glittering Hapsburg court. And it had been dropped with deliberate accuracy on his own part of Europe--that frail and volatile part--the part about to blow up.

The European kiosk was an almighty vending machine. It replaced its rapidly dwindling stocks in the Black Maria middle of the night, with unmanned cargo vehicles, flat blind anonymous cockroach-like robot things of pink plastic and pink rubber wheels, that snuffled and radared their way across the midnight city and obeyed every traffic law with a crazy punctiliousness.
There was no one to talk to inside the pink European kiosk, although, when addressed through its dozens of microphones, the kiosk could talk the local language, rather beautifully. There were no human relations to be found there. There was no such thing as society: only a crisp interaction.

Gangs of kids graffiti-tagged the pink invader right away. Someone--Ace most likely--made a serious effort to burn it down.

They found Ace dead two days later,  in his fancy electric sports car, with three fabbed black bullets through him, and a fabbed black pistol abandoned on the car's hood.

\vfil
\centerline{{$\bigtriangleup\bigtriangledown\bigtriangleup\bigtriangledown\bigtriangleup$}}

\chapter*{\centerline{VII}}

Ivana caught him before he could leave for the hills.

``You would go without a word, wouldn't you? Not one word to me, and again you just go!''

``It's the time to go.''

``You'd take crazy students with you. You'd take football bullies. You'd take tough-guy gangsters. You'd take gypsies and crooks. You'd go there with anybody. And not take me?''

``We're not on a picnic. And you're not the kind of scum who goes to the hills when there's trouble.''

``You're taking guns?''

``You women never understand! You don't take carbines with you when you've got a black factory that can make carbines!'' Borislav rubbed his unshaven jaw. Ammo, yes, some ammo might well be needed.

Grenades, mortar rounds. He knew all too well how much of that stuff had been buried out in the hills, since the last time. It was like hunting for truffles.

And the land mines. Those were what really terrified him, in an unappeasable fear he would take to his grave. Coming back toward the border, once, he and his fellow vigilantes, laden with their loot, marching in step in the deep snow, each man tramping in another's sunken boot-prints\dots Then a flat, lethal thing, with a chip, a wad of explosive and a bellyful of steel bolts, counted their passing footsteps. The virgin snow went blood-red.

Borislav might have easily built such a thing himself. The shade-tree plans for such guerrilla devices were everywhere on the net. He had never built such a bomb, though the prospect gnawed him in nightmares.

Crippled for life, back then, he had raved with high fever, freezing, starving, in a hidden village in the hills. His last confidante was his nurse. Not a wife, not a lover, not anyone from any army, or any gang, or any government. His mother. His mother had the only tie to him so profound that she would leave her city, leave everything, and risk starvation to look after a wounded guerrilla. She brought him soup. He watched her cheeks sink in day by day as she starved herself to feed him.

``You don't have anyone to cook for you out there,'' Ivana begged.

``You'd be leaving your daughter.''

``You're leaving your mother.''

She had always been able to sting him that way. Once again, despite everything he knew, he surrendered. ``All right, then,'' he told her. ``Fine. Be that way, since you want it so much. If you want to risk everything, then you can be our courier. You go to the camp, and you go to the city. You carry some things for us. You never ask any questions about the things.''

``I never ask questions,'' she lied. They went to the camp and she just stayed with him. She never left his side, not for a day or a night. Real life started all over for them, once again. Real life was a terrible business.

\vfil\centerline{{$\bigtriangleup\bigtriangledown\bigtriangleup\bigtriangledown\bigtriangleup$}}

\chapter*{\centerline{VIII}}

It no longer snowed much in the old ski villages; the weather was a real mess nowadays, and it was the summers you had to look out for. They set up their outlaw fab plant inside an abandoned set of horse stables.

The zealots talked wildly about copying an ``infinite number of fabs,'' but that was all talk. That wasn't needed. It was only necessary to make and distribute enough fabs to shatter the nerves of the authorities. That was propaganda of the deed.

Certain members of the government were already nodding and winking at their efforts. That was the only reason that they might win. Those hustlers knew that if the weathervanes spun fast enough, the Byzantine cliques that ruled the statehouse would have to break up. There would be chaos. Serious chaos.

But then, after some interval, the dust would have to settle on a new arrangement of power-players.

Yesterday's staunch conservative, if he survived, would become the solid backer of the new regime. That was how it worked in these parts.

In the meantime, however, some dedicated group of damned fools would have to actually carry out the campaign on the ground. Out of any ten people willing to do this, seven were idiots. These seven were dreamers, rebels by nature, unfit to run so much as a lemonade stand.
One out of the ten would be capable and serious. Another would be genuinely dangerous: a true, amoral fanatic. The last would be the traitor to the group: the police agent, the coward, the informant.

There were thirty people actively involved in the conspiracy, which naturally meant twenty-one idiots. 

Knowing what he did, Borislav had gone there to prevent the idiots from quarrelling over nothing and blowing the effort apart before it could even start. The three capable men had to be kept focussed on building the fabs. The fanatics were best used to sway and intimidate the potential informants. If they held the rebellion together long enough, they would wear down all the sane people. That was the victory.

The rest was all details, where the devil lived. The idea of self-copying fabs looked great on a sheet of graph paper, but it made little practical sense to make fabs entirely with fabs. Worse yet, there were two vital parts of the fab that simply couldn't be fabbed at all. One was the nozzle that integrated the yellow dust into the black stuff. The other was the big recycler comb that chewed up the black stuff back into the yellow dust. These two crucial components obviously couldn't be made of the yellow dust or the black stuff.

Instead, they were made of precisely machined high-voltage European metals that were now being guarded like jewels. These components were way beyond the conspiracy's ability to create.
Two dozen of the fabbing nozzles showed up anyway. They came through the courtesy of some foreign intelligence service. Rumor said the Japanese, for whatever inscrutable reason.
They still had no recycling combs. That was bad. It confounded and betrayed the whole dream of fabs to make them with the nozzles but not the recycling combs. This meant that their outlaw fabs could make things, but never recycle them. A world with fabs like that would be a nightmare: it would slowly but surely fill up with horrible, polluting fabjunk: unusable, indestructible, rock-solid lumps of black slag.

Clearly this dark prospect had much affected the counsels of the original inventors. There were also many dark claims that carbon nanotubes had dire health effects: because they were indestructible fibers, something like asbestos. And that was true: carbon nanotubes did cause cancer. However, they caused rather less cancer than several thousand other substances already in daily use. It took all summer for the competent men to bang together the first outlaw fabs. Then it became necessary to sacrifice the idiots, in order to distribute the hardware. The idiots, shrill and eager as ever, were told to drive the fabs as far as possible from the original factory, then hand them over to sympathizers and scram. Four of the five idiots were arrested almost at once. Then the camp was raided by helicopters.

However, Borislav had fully expected this response. He had moved the camp. In the city, riots were under way. It didn't matter who ``won'' these riots, because rioting melted the status quo. The police were hitting the students with indestructible black batons. The kids were slashing their paddy-wagon tires with indestructible black kitchen knives.

At this point, one of the fanatics had a major brainwave. He demanded that they send out dozens of fake black boxes that merely looked like fabs. There was no political need for their futuristic promises of plenty to actually work. This cynical scheme was much less work than creating real fabs, so it was swiftly adopted. More than that: it was picked up, everywhere, by copycats. People were watching the struggle: in Bucharest, Lublin, Tbilisi; in Bratislava, Warsaw, and Prague. People were dipping ordinary objects in black lacquer to make them look fabbed. People were distributing handbooks for fabs, and files for making fabs. For every active crank who really wanted to make a fab, there were a hundred people who wanted to know how to do it. Just in case.

Some active cranks were succeeding. Those who failed became martyrs. As resistance spread like spilled ink, there were simply too many people implicated to classify it as criminal activity. 
Once the military contractors realized there were very good reasons to make giant fabs the size of shipyards, the game was basically over. Transition Three was the new realpolitik. The new economy was the stuff of the everyday. The older order was over. It was something no one managed to remember, or even wanted to manage to remember.

The rest of it was quiet moves toward checkmate. And then the game just stopped. Someone tipped over the White King, in such a sweet, subtle, velvety way that one would have scarcely guessed that there had ever been a White King to fight against at all.

\vfil
\centerline{$\bigtriangleup\bigtriangledown\bigtriangleup\bigtriangledown\bigtriangleup$}

\chapter*{\centerline{IX}}

Borislav went to prison. It was necessary that somebody should go. The idiots were only the idiots.

The competent guys had quickly found good positions in the new regime. The fanatics had despaired of the new dispensation, and run off to nurse their bitter disillusionment.
As a working rebel whose primary job had been public figurehead, Borislav was the reasonable party for public punishment.

Borislav turned himself in to a sympathetic set of cops who would look much better for catching him.

They arrested him in a blaze of publicity. He was charged with ``conspiracy'': a rather merciful charge, given the host of genuine crimes committed by his group. Those were the necessary, everyday crimes of any revolution movement, crimes such as racketeering, theft of services, cross-border smuggling, subversion and sedition, product piracy, copyright infringement, money laundering, fake identities, squatting inside stolen property, illegal possession of firearms, and so forth.

Borislav and his various allies weren't charged with those many crimes. On the contrary; since he himself had been so loudly and publicly apprehended, those crimes of the others were quietly overlooked.

While sitting inside his prison cell, which was not entirely unlike a kiosk, Borislav discovered the true meaning of the old term ``penitentiary.'' The original intention of prisons was that people inside them should be penitent people. Penitent people were supposed to meditate and contemplate their way out of their own moral failings. That was the original idea.
Of course, any real, modern ``penitentiary'' consisted mostly of frantic business dealings. Nobody ``owned'' much of anything inside the prison, other than a steel bunk and a chance at a shower, so simple goods such as talcum powder loomed very large in the local imagination. Borislav, who fully understood street-trading, naturally did very well at this. At least, he did much better than the vengeful, mentally limited people who were doomed to inhabit most jails. Borislav thought a lot about the people in the jails.

They, too, were the people, and many of those people were getting into jail because of him. In any Transition, people lost their jobs. They were broke, they lacked prospects. So they did something desperate.

Borislav did not much regret the turmoil he had caused the world, but he often thought about what it meant and how it must feel. Somewhere, inside some prison, was some rather nice young guy, with a wife and kids, whose job was gone because the fabs took it away. This guy had a shaven head, an ugly orange jumpsuit, and appalling food, just like Borislav himself. But that young guy was in the jail with less good reason. And with much less hope. And with much more regret. That guy was suffering. Nobody gave a damn about him. If there was any justice, someone should mindfully suffer, and be penitent, because of the harsh wrong done that guy.
Borislav's mother came to visit him in the jail. She brought printouts from many self-appointed sympathizers. The world seemed to be full of strange foreign people who had nothing better to do with their time than to e-mail tender, supportive screeds to political prisoners. Ivana, something of a mixed comfort to him in their days of real life, did not visit the jail or see him. Ivana knew how to cut her losses when her men deliberately left her to do something stupid, such as volunteering for a prison. 

These strangers and foreigners expressed odd, truncated, malformed ideas of what he had been doing.

Because they were the Voice of History. He himself had no such voice to give to history. He came from a small place under unique circumstances. People who hadn't lived there would never understand it. Those who had lived there were too close to understand it. There was just no understanding for it. There were just ... the events. Events, transitions, new things. Things like the black kiosks.

These new kiosks\dots No matter where they were scattered in the world, they all had the sinister, strange, overly dignified look of his own original black kiosk. Because the people had seen those kiosks. The people knew well what a black fabbing kiosk was supposed to look like. Those frills, those fringes, that peaked top, that was just how you knew one. That was their proper look. You went there to make your kid's baby shoes indestructible. The kiosks did what they did, and they were what they were. They were everywhere, and that was that. After twenty-two months, a decent interval, the new regime pardoned him as part of a general amnesty. He was told to keep his nose clean and his mouth shut. Borislav did this. He didn't have much to say, anyway.

\vfil
\centerline{$\bigtriangleup\bigtriangledown\bigtriangleup\bigtriangledown\bigtriangleup$}

\chapter*{\centerline{X}}

Time passed. Borislav went back to the older kind of kiosk. Unlike the fancy new black fabbing kiosk, these older ones sold things that couldn't be fabbed: foodstuffs, mostly.

Now that fabs were everywhere and in public, fabbing technology was advancing by leaps and bounds.

Surfaces were roughened so they shone with pastel colors. Technicians learned how to make the fibers fluffier, for bendable, flexible parts. The world was in a Transition, but no transition ended the world. A revolution just turned a layer in the compost heap of history, compressing that which now lay buried, bringing air and light to something hidden.

On a whim, Borislav went into surgery and had his shinbone fabbed. His new right shinbone was the identical, mirror-reversed copy of his left shinbone. After a boring recuperation, for he was an older man now and the flesh didn't heal as it once had, he found himself able to walk on an even keel for the first time in twenty-five years.

Now he could walk. So he walked a great deal. He didn't skip and jump for joy, but he rather enjoyed walking properly. He strolled the boulevards, he saw some sights, he wore much nicer shoes.

Then his right knee gave out, mostly from all that walking on an indestructible artificial bone. So he had to go back to the cane once again. No cure was a miracle panacea: but thanks to technology, the trouble had crept closer to his heart. That made a difference. The shattered leg had oppressed him during most of his lifetime. That wound had squeezed his soul into its own shape. The bad knee would never have a chance to do that, because he simply wouldn't live that long. So the leg was a tragedy. The knee was an episode.

It was no great effort to walk the modest distance from his apartment block to his mother's grave. The city kept threatening to demolish his old apartments. They were ugly and increasingly old-fashioned, and they frankly needed to go. But the government's threats of improvement were generally empty, and the rents would see him through. He was a landlord. That was never a popular job, but someone was always going to take it. It might as well be someone who understood the plumbing. It gave him great satisfaction that his mother had the last true granite headstone in the local graveyard. All the rest of them were fabbed.

Dr. Grootjans was no longer working in a government. Dr. Grootjans was remarkably well-preserved. If anything, this female functionary from an alien system looked younger than she had looked, years before.

She had two prim Nordic braids. She wore a dainty little off-pink sweater. She had high heels. Dr. Grootjans was writing about her experiences in the transition. This was her personal, confessional text, on the net of course, accompanied by photographs, sound recordings, links to other sites, and much supportive reader commentary.

``Her gravestone has a handsome Cyrillic font,'' said Dr. Grootjans. Borislav touched a handkerchief to his lips. ``Tradition does not mean that the living are dead. Tradition means that the dead are living.''

Dr. Grootjans happily wrote this down. This customary action of hers had irritated him at first.
However, her strange habits were growing on him. Would it kill him that this overeducated foreign woman subjected him to her academic study?

Nobody else was bothering. To the neighborhood, to the people, he was a crippled, short-tempered old landlord. To her, the scholar-bureaucrat, he was a mysterious figure of international significance. Her version of events was hopelessly distorted and self-serving. But it was a version of events.

``Tell me about this grave,'' she said. ``What are we doing here?''

``You wanted to see what I do these days. Well, this is what I do.'' Borislav set a pretty funeral bouquet against the headstone. Then he lit candles.

``Why do you do this?''

``Why do you ask?''

``You're a rational man. You can't believe in religious rituals.''

``No,'' he told her, ``I don't believe. I know they are just rituals.''

``Why do it, then?''

He knew why, but he did not know how to give her that sermon. He did it because it was a gift. It was a liberating gift for him, because it was given with no thought of any profit or return. A deliberate gift with no possibility of return. Those gifts were the stuff of history and futurity. Because gifts of that kind were also the gifts that the living received from the dead. The gifts we received from the dead: those were the world's only genuine gifts. All the other things in the world were commodities. The dead were, by definition, those who gave to us without reward. And, especially: our dead gave to us, the living, within a dead context. Their gifts to us were not just abjectly generous, but archaic and profoundly confusing.

Whenever we disciplined ourselves, and sacrificed ourselves, in some vague hope of benefiting posterity, in some ambition to create a better future beyond our own moment in time, then we were doing something beyond a rational analysis. Those in that future could never see us with our own eyes: they would only see us with the eyes that we ourselves gave to them. Never with our own eyes: always with their own. And the future's eyes always saw the truths of the past as blinkered, backward, halting. 

Superstition.

``Why?'' she said.

Borislav knocked the snow from his elegant shoes. ``I have a big heart.''

\vfil
\centerline{$\bigcirc$}

\end{document}
